\section{QA-Prolog}
\begin{frame}{QA-Prolog}
	\begin{minipage}{.48\linewidth}
		\begin{block}{Il progetto}
			\begin{itemize}
				\item Sviluppato da Sott Pakin
				\item Proof of concept
				\item Trasformazioni successive
				\item Prolog + Query $\rightarrow \mathcal{H}_f$  
				\item Interagisce direttamente con il solver
				\item Raccoglie e organizza i risultati
			\end{itemize}
		\end{block}
	\end{minipage}
	\hfill
	\begin{minipage}{.48\linewidth}
		immagine
	\end{minipage}
\end{frame}


\begin{frame}{Pipeline}
	\begin{minipage}{.48\linewidth}
		immagine
	\end{minipage}
	\hfill
	\begin{minipage}{.48\linewidth}
		\begin{block}{Trasformazioni}
			\begin{enumerate}
				\item Prolog $\rightarrow$ Verilog (HDL)
				\item Verilog $\rightarrow$ Circuito digitale
				\item Circuito digitale $\rightarrow$ $\mathcal{H}_f$ simbolica 
				\item $\mathcal{H}_f$ simbolica  $\rightarrow$ $\mathcal{H}_f$ fisica 
			\end{enumerate}
		\end{block}
	\end{minipage}
\end{frame}


\begin{frame}{QMASM}{$\mathcal{H}_f$ simbolica  $\rightarrow$ $\mathcal{H}_f$ fisica}
		\begin{minipage}{.48\linewidth}
			\begin{block}{Cos'è}
				\begin{itemize}
					\item Quantum macro assembler
					\item Sviluppato in Python
					\item Basso livello di astrazione
					\item Si interfaccia con Ocean
				\end{itemize}
			\end{block}
		\end{minipage}
		\hfill
		\begin{minipage}{.48\linewidth}
			\begin{block}{Cosa permette di fare}
				\begin{itemize}
					\item Riferimento simbolico a \emph{qubit}
					\item \emph{Qubit} \say{pinnati} a \texttt{TRUE} o \texttt{FALSE}
					\item Incapsulare pattern in macro
					\item Creazione di librerie di macro
					\item Pulizia dell'output:
					\begin{itemize}
						\item solo \emph{qubit} \say{interessanti}
						\item no slack variables
					\end{itemize}
				\end{itemize}
			\end{block}
		\end{minipage}
\end{frame}


\begin{frame}[fragile]{QMASM}{Esempio: Macro}
	\begin{minipage}{.3\linewidth}
		\begin{figure}[h]
			\begin{minipage}{\linewidth}
				\hrule
				\begin{minted}{text}
					
# Y = A OR B
!begin_macro OR
  $A  0.5
  $B  0.5
  $Y -1

  $A $B  0.5
  $A $Y -1
  $B $Y -1
!end_macro OR
				\end{minted}
				\hrule
			\end{minipage}
			\caption{\textbf{or} gate}
		\end{figure}
	\end{minipage}
	\hfill
	\begin{minipage}{.3\linewidth}
		\begin{figure}[h]
			\begin{minipage}{\linewidth}
				\hrule
				\begin{minted}{text}

# Y = NOT A
!begin_macro NOT
  $A $Y 1.0
!end_macro NOT
				\end{minted}
				\hrule
			\end{minipage}
			\caption{\textbf{not} gate}
		\end{figure}
	\end{minipage}
	\hfill
	\begin{minipage}{.3\linewidth}
		\begin{figure}[h]
			\begin{minipage}{\linewidth}
				\hrule
				\begin{minted}{text}
					
# Y = A AND B
!begin_macro AND
  $A -0.5
  $B -0.5
  $Y  1

  $A $B  0.5
  $A $Y -1
  $B $Y -1
!end_macro AND
				\end{minted}
				\hrule
			\end{minipage}
			\caption{\textbf{and} gate}
		\end{figure}
	\end{minipage}
	\begin{block}{}
		\centering
		possiamo racchiudere queste macro nel file \texttt{gates.qmasm}
	\end{block}
\end{frame}

\begin{frame}[fragile]{QMASM}{Esempio: $y = x_1 \land \neg(x_2 \lor x_3)$}
	\begin{center}
	\begin{minipage}{.25\linewidth}
		\begin{figure}[h]
			\begin{minipage}{\linewidth}
				\hrule
				\begin{minted}{text}
					
!include <gates>

!use_macro OR x2_or_x3
x2_or_x3.$A = x2
x2_or_x3.$B = x3
x2_or_x3.$Y = $x4

!use_macro NOT not_x4
not_x4.$A = $x4
not_x4.$Y = $x5

!use_macro AND x1_and_x5
x1_and_x5.$A = x1
x1_and_x5.$B = $x5
x1_and_x5.$Y = y
				\end{minted}
				\hrule
			\end{minipage}
			\caption{CircSat problem}
		\end{figure}
	\end{minipage}
	\hspace{1cm}
	\begin{minipage}{.6\linewidth}
		\begin{block}{}
			\say{Pinniamo} il valore di $y$ per ottenere l'assegnamento delle $x_i$ che verificano la formula logica:
			\begin{center}
				\scriptsize \texttt{qmasm --run --pin="y := true" circsat.qmasm}
			\end{center}
		\end{block}
		\begin{figure}[h]
			\centering
			\begin{minipage}{.72\linewidth}
				\hrule
				\begin{minted}{text}
					
Solution #1 (energy = -20.0000, tally = 647):

    Variable  Value
    --------  -----
    x1        True
    x2        False
    x3        False
    y         True
				\end{minted}
				\hrule
			\end{minipage}
			\caption{CircSat solution}
		\end{figure}
	\end{minipage}
	\end{center}
\end{frame}


\begin{frame}{Yosys - edif2qmasm}
	\begin{block}{}
		\centering
		Verilog $\rightarrow$ Circuito digitale $\rightarrow \mathcal{H}$ simbolica 
	\end{block}
	\begin{minipage}{.48\linewidth}
		\begin{block}{Yosys}
			\begin{itemize}
				\item Framework per la sintesi del Verilog
				\item Free and open software sotto licenza ISC
				\item Output: RTL Netlist in formato EDIF
			\end{itemize}
		\end{block}
		Immagine
	\end{minipage}
	\hfill
	\begin{minipage}{.48\linewidth}
		\begin{block}{edif2qmasm}
			\begin{itemize}
				\item Converte dal formato EDIF a QMASM
				\item Attinge a una libreria di gate
			\end{itemize}
		\end{block}
		Immagine
	\end{minipage}
\end{frame}


\begin{frame}[fragile]{Yosys - edif2qmasm}{Esempio: moltiplicazione tra interi}
		\begin{minipage}{.48\linewidth}
			\begin{figure}[h]
				\begin{minipage}{\linewidth}
					\hrule
					\begin{minted}{Verilog}
					
module mult (multiplicand, multiplier, product);
   input [1:0] multiplicand;
   input [1:0] multiplier;
   output[2:0] product;

   assign product = multiplicand * multiplier;
endmodule
					\end{minted}
					\hrule
				\end{minipage}
			\caption{Factorization problem}
		\end{figure}
		\vspace{-.7cm}
		\begin{block}{}
			Tradotto in EDIF con:
			\begin{center}
				\scriptsize \texttt{yosys myfile.v synth.ys -b edif -o myfile.edif}
			\end{center}
		\end{block}
		\begin{block}{}
			Tradotto in QMASM con:
			\begin{center}
				\scriptsize \texttt{edif2qmasm -o="myfile.qmasm" myfile.edif}
			\end{center}
		\end{block}
	\end{minipage}
	\hfill
	\begin{minipage}{.48\linewidth}
		\begin{block}{}
			Eseguito  con:\\
			\begin{center}
				\scriptsize \texttt{qmasm --run --pin="mult.product[2:0] := 110" --solver="sim\_anneal" mult.qmasm}
			\end{center}
		\end{block}
			\begin{figure}[h]
				\begin{minipage}{.9\linewidth}
					\hrule
					\begin{minted}{text}
					
Solution #1 (energy = -57.5000, tally = 68):

    Variable              Value
    --------------------  -----
    mult.multiplicand[0]  False
    mult.multiplicand[1]  True
    mult.multiplier[0]    True
    mult.multiplier[1]    True
    mult.product[0]       False
    mult.product[1]       True
    mult.product[2]       True
					\end{minted}
					\hrule
				\end{minipage}
			\caption{Factorization Solution}
		\end{figure}
	\end{minipage}
\end{frame}


\begin{frame}{QA-Prolog}
	\begin{minipage}{.48\linewidth}
		\begin{block}{}
			\begin{itemize}
				\item Traduzione da Prolog a Verilog
				\item Wrapper per tutta la Pipeline
				\item Risultati in formato Human Readable
				\item Decide dimensione delle variabili
			\end{itemize}
		\end{block}
	\end{minipage}
	\hfill
	\begin{minipage}{.48\linewidth}
		immagine
	\end{minipage}
\end{frame}


\begin{frame}{QA-Prolog}{Esempio}
	
\end{frame}
